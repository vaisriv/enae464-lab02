%%%%%%%%%%%%%%%%%%%%%%%%%
% Imports and Setup     %
%%%%%%%%%%%%%%%%%%%%%%%%%

\input{../lib/hw-setup/tex/HWSetup}
\input{../lib/hw-setup/tex/EngBindings}
\usetikzlibrary{arrows.meta, decorations.markings, patterns, calc, positioning}

\fancyhead[C]{}
\fancyhead[R]{\hyperlink{tableofcontents}{\hmwkClass\ (\hmwkClassInstructor,\ \hmwkClassTime): \hmwkTitle}}
\fancyheadwidth[L]{0.5\headwidth}
\fancyheadwidth[C]{0.0\headwidth}
\fancyheadwidth[R]{0.5\headwidth}

%%%%%%%%%%%%%%%%%%%%%%%%%
% Homework Details      %
% - Title               %
% - Subtitle            %
% - Due Date            %
% - Due Time            %
% - Course              %
% - Section Time        %
% - Instructor          %
% - Author              %
% - Submission Time     %
%%%%%%%%%%%%%%%%%%%%%%%%%

\hwkTitle{Lab02}
\hwkSubTitle{Pressure Drag and Lift on an Airfoil Model}
\hwkDueDate{2026-02-27}
\hwkDueTime{23:59:00}
\hwkClass{ENAE464 - 0202}
\hwkClassTime{14:00:00}
\hwkInstructor{Dr. Silbaugh}
\hwkAuthor{Mikołaj Kostrzewa \& Vai Srivastava}
\hwkCompletionDate{
        Experiment Performed: \DTMdate{2026-02-20}\\
        Report Submitted: \DTMtoday
}

\addbibresource{./references.bib}

\begin{document}

\hypertarget{titlepage}{\maketitle}
\thispagestyle{empty}

\pagebreak

%%%%%%%%%%%%%%%%%%%%%%%%%
% Letter of Transmittal %
%%%%%%%%%%%%%%%%%%%%%%%%%

\thispagestyle{empty}

\DTMtoday

\vspace{5.0ex}

Enclosed is the technical report for the \textit{Pressure Drag and Lift on an Airfoil Model} laboratory experiment. This report presents the experimental methodology, results, and analysis of aerodynamic characteristics observed on the surface of a scale model of a NACA airfoil, including lift coefficient and drag coefficient distributions, aerodynamic force analyses, and comparison with published NACA data.

Please feel free to contact us with any questions regarding the contents of this report.

\vspace{5.0ex}

Respectfully,

Mikołaj Kostrzewa \& Vai Srivastava

%%%%%%%%%%%%%%%%%%%%%%%%%
% Table of Contents     %
%%%%%%%%%%%%%%%%%%%%%%%%%

\pagebreak

\thispagestyle{empty}
\hypertarget{tableofcontents}{\tableofcontents}

\renewcommand{\listfigurename}{Figures}
\listoffigures

\renewcommand{\listtablename}{Tables}
\listoftables

\renewcommand{\listlistingname}{Listings}
\listoflistings

\newpage
\pagenumbering{arabic}

%%%%%%%%%%%%%%%%%%%%%%%%%
% Report                %
%%%%%%%%%%%%%%%%%%%%%%%%%

\section{Abstract} \label{sec:abstract}

%FIXME: fill in section
\lipsum[1-2]

\section{Introduction} \label{sec:introduction}

%FIXME: fill in section
\lipsum[1-2]

\section{Experimental Apparatus and Procedures} \label{sec:procedures}

\subsection{Experimental Apparatus} \label{sec:procedures/apparatus}

%FIXME: fill in section
\lipsum[1-2]

\subsection{Operating Technique} \label{sec:procedures/technique}

%FIXME: fill in section
\lipsum[1-2]

\section{Results} \label{sec:results}

\subsection{Operating Conditions} \label{sec:results/conditions}

\begin{table}[H]
    \centering
    \caption{Ambient Room Conditions during Experiment}
    \begin{tabular}{|l|c|c|} \hline
        Quantity                & Value              & Uncertainty                          \\ \hline \hline
        Room Temperature, \(T\) & \qty{25}{\c}       & \(\pm \qty{0.5}{\c}\)                \\
        Room Pressure, \(P\)    & \qty{1009.0}{\hecto\Pa} & \( \pm \qty{0.05}{\hecto\Pa} \) \\ \hline
    \end{tabular}
    \label{tab:ambient-conditions}
\end{table}

The density of air (\(\rho\)) can be calculated using the Ideal Gas Law:
\begin{equation}
    \rho = \frac{P}{R T}
\end{equation}
where:
\begin{itemize}
    \item \(P\) is the absolute pressure (in SI units: Pascals \unit{\Pa})
    \item \(T\) is the absolute temperature (in Kelvin \unit{\K})
    \item \(R\) is the specific gas constant for dry air, \(R = \qty{287.05}{\J\per\kg\per\K}\)
\end{itemize}

Given measurements:
\begin{align}
    P      &= \qty{1009.0}{\hecto\Pa} = \qty{100900}{\Pa} \\
    T      &= \qty{25}{\C} = 25 + 273.15 = \qty{298.15}{\K} \\
    R      &= \qty{287.05}{\J\per\kg\per\K}
\end{align}

\begin{equation}
    \rho = \frac{\qty{100900}{\Pa}}{\qty{287.05}{\J\per\kg\per\K}} \times \qty{298.15}{\K}
\end{equation}

Calculate:
\begin{align}
    \rho &= \frac{100900}{287.05 \times 298.15} \\
         &= \frac{100900}{85542.5} \\
         &\approx \qty{1.18}{\kg\per\m\cubed}
\end{align}

Thus, \textbf{the ambient air density during the experiment is}:
\begin{equation}
    \boxed{\rho = \qty{1.18}{\kg\per\m\cubed}}
\end{equation}

\subsection{Measurements} \label{sec:results/measurements}

% TODO:
% A table of the pressures at each pitot tap, \( P_{n} \), over the airfoil on both upper and lower surfaces for each angle of attack. (One table with multiple columns of data.) See table below for surface coordinates (as well as tap locations).

%FIXME: fill in section
\lipsum[1-2]

\subsection{Quantities of Interest} \label{sec:results/quantities}

\subsubsection{Incoming Wind Tunnel Airflow} \label{sec:results/quantities/incoming-airflow}
% TODO:
% Calculations of the incoming wind velocity \( \left(U_{\infty}\right) \) and the Reynolds number for the airfoil flow.

%FIXME: fill in section
\lipsum[1-2]

\subsubsection{Coefficient of Pressure} \label{sec:results/quantities/Cp}

% TODO:
% A table of the pressure coefficient, \( C_{p}=2\left(P-P_{\infty}\right) / \rho U_{\infty}^2 \), over the airfoil on both upper and lower surfaces for each angle of attack. (One table with multiple columns of data.)

% TODO:
% A plot of \( C_{p} \) on both upper and under surfaces versus non-dimensional distance, \(x / c\), for each angle of attack. Your plot should also include a plot of the airfoil surface. See table below for surface coordinates (as well as tap locations).

%FIXME: fill in section
\lipsum[1-2]

\subsubsection{Pressure Drag and Lift Force} \label{sec:results/quantities/pressure-forces}

% TODO:
% An estimate of the pressure drag and lift force at each angle of attack.

%FIXME: fill in section
\lipsum[1-2]

\subsubsection{Coefficient of Lift} \label{sec:results/quantities/CL}

% TODO:
% One plot of lift coefficient, \( C_{L} = 2F_{L} / \rho U_{\infty}^{2} \), versus the angle of attack.

% TODO:
% An estimate of the maximum value of \( C_{L} \) and the angle of attack at which the airfoil stalls.

%FIXME: fill in section
\lipsum[1-2]

\subsubsection{Coefficient of Drag} \label{sec:results/quantities/CD}

% TODO:
% One plot of lift coefficient, \( C_{D} = 2F_{D} / \rho U_{\infty}^{2} \), versus the angle of attack.

%FIXME: fill in section
\lipsum[1-2]

\section{Analysis and Discussion} \label{sec:analysis}

\subsection{Experimental Aerodynamic Characteristics vs. Published NACA Data} \label{sec:analysis/CL-vs-naca}

% TODO:
% Compare max \( C_{L} \) and max angle of attack to published NACA data for this airfoil. Do you get good correlation? Explain any discrepancies.

%FIXME: fill in section
\lipsum[1-2]

\subsection{Experimental Lift Curve vs. Published NACA Data} \label{sec:analysis/lift-curve-vs-naca}

% TODO:
% Compare the lift curve slope with published NACA data. Do you get good correlation? Explain any discrepancies.

%FIXME: fill in section
\lipsum[1-2]

\subsection{Experimental Drag Polar vs. Published NACA Data} \label{sec:analysis/drag-polar-vs-naca}

% TODO:
% Compare your drag polar with published NACA data. Do you capture the same trends? Are there any regions where they disagree? Explain any discrepancies.

%FIXME: fill in section
\lipsum[1-2]

\section{Summary and Conclusions} \label{sec:summary}

%FIXME: fill in section
\lipsum[1-2]

\section{References} \label{sec:references}

\printbibliography[heading=none]

\section{Appendices} \label{sec:appendices}

\subsection{Data} \label{sec:appendices/data}

% \begin{table}[H]
%     \centering
%     \caption{Raw Measurements of Pressure Taps vs. Angle of Attack}
%     \mintinline{bash}{./data/pressure_vs_theta.csv} \\
%     \begin{tabular}[c]{|r|l|c|c|c|}\hline
%         & \( \theta \, \left[\unit{\deg}\right] \) & \( P_{\infty} \, \left[\unit{\cm}\right] \) & \( P_{0} \, \left[\unit{\cm}\right] \) & \( P \, \left[\unit{\cm}\right] \) \\ \hline \hline
%         \csvreader[late after line = \\ \hline]{../data/pressure_vs_theta.csv}{}{
%             {\setpadnum{2}\tt\scriptsize\padnum{\thecsvrow}} &
%             \csvcoli &
%             \csvcolii &
%             \csvcoliii &
%             \csvcoliv
%         }
%     \end{tabular}
%     \label{tab:data/pressure}
% \end{table}

\begin{table}[H]
    \centering
    \caption{NACA 4412 Airfoil Coordinates and the Corresponding Tap Locations on the Model Airfoil \cite{mselig2026uiuc}}
    \mintinline{bash}{./data/naca_4412_airfoil_coords_and_taps.csv} \\
    \begin{tabular}[c]{|l|l|c|c|} \hline
        \( x/c \) & \( y/c \) & Tap \# & DSA Channel \# \\ \hline \hline
        \csvreader[late after line = \\ \hline]{../data/naca_4412_airfoil_coords_and_taps.csv}{}{
            \csvcoli &
            \csvcolii &
            \IfCsvsimStrEqualTF{\csvcoliii}{-}{--}{{\setpadnum{2}\padnum{\csvcoliii}}} &
            \IfCsvsimStrEqualTF{\csvcoliv}{-}{--}{{\setpadnum{2}\padnum{\csvcoliv}}}
        }
    \end{tabular}
    \label{tab:data/naca-4412-model-coords}
\end{table}

\begin{table}[H]
    \centering
    \caption{NACA 4412 Airfoil Drag Polar \cite{mdrela2026xfoil}}
    \mintinline{bash}{./data/xf-naca4412-il-1000000.csv} \\
    \inputminted{text}{../data/xf-naca4412-il-1000000.txt}
    \label{tab:data/naca-4412-drag-polar}
\end{table}
\begin{longtable}{|r|c|c|c|c|c|c|} \hline
    \( \alpha \) & \( C_{l} \) & \( C_{d} \) & \( C_{dp} \) & \( C_{m} \) & \( \mathrm{Top}_{\mathrm{Xtr}} \) & \( \mathrm{Bot}_{\mathrm{Xtr}} \) \\ \hline \hline
    \csvreader[
        late after line = \\ \hline
    ]{../data/xf-naca4412-il-1000000.csv}{}{
        \csvcoli &
        \csvcolii &
        \csvcoliii &
        \csvcoliv &
        \csvcolv &
        \csvcolvi &
        \csvcolvii
    }
\end{longtable}

\subsection{Code} \label{sec:appendices/code}

For the sake of brevity, only the code files that are key to the analysis are included below. However, in the spirit of completeness, the repository containing the complete data, source code, and notes for this report can be found at \href{https://www.github.com/vaisriv/enae464-lab02}{github:vaisriv/enae464-lab02}.

\begin{codeblock}
    \centering
    \caption{Index File}
    \mintinline{bash}{./src/index.py}
    \inputminted{python}{../src/index.py}
    \label{lst:code/index}
\end{codeblock}

% render all references
\nocite{*}

\end{document}
